
\documentclass{article}
\usepackage[utf8]{inputenc} %en español
\usepackage{graphicx} %fotos
\usepackage{multicol}
\graphicspath{ {./} }
\usepackage{indentfirst}
\usepackage{array}
\usepackage{float}

\title{Actividad 1} 
\author{Luis Eduardo Martínez Espinoza \\
Departamento de Física \\
Universidad de Sonora} 

\date{15 de enero de 2021} 
 
\begin{document}  
\maketitle 


\section{Introducción}
En este trabajo se presentera información del clima en Tecate, Baja California, toda esta obtenida de la base de datos de CONAGUA. Se presentaran gráficas y un poco de información acerca de este municipio. Estoy interesado específicamente por este lugar porque tengo amistades que residen ahí y seguido discutimos sobre la diferencia entre los climas de Tecate y Hermosillo, Sonora, que es donde me encuentro actualmente. Con este trabajo mi objetivo es conocer como es el clima de allá. 

\section{Descripción}
\noindent En la tabla a continuación se muestra información sobre la estación elegida: \\

\begin{center}
\begin{tabular}{ |p{3cm}|c|}
 \hline
 \textbf{Número}& 2030\\
  \hline
 \textbf{Nombre}& La Puerta\\
  \hline
 \textbf{Latitud}& 32.5439º\\
  \hline
 \textbf{Longitud}& -116.6783º\\
  \hline
 \textbf{Altura sobre el nivel del mar}& 480 m\\
  \hline
 \textbf{Rango de años de datos disponibles}& 66\\
 \hline
\end{tabular}
\end{center}

\\En la Figura 1 se muestra una vista aérea de la ciudad y en la Figura 2 se muestra la ubicación geográfica.\\ \\ 

\begin{figure}[H]
    \centering
    \includegraphics[height=8cm]{a.jpeg}
    \caption{Vista aérea}
    \label{fig:my_label}
\end{figure}
\begin{figure}[H]
    \centering
    \includegraphics[height=8cm]{b.png}
    \caption{Locación geográfica}
    \label{fig:my_label}
\end{figure}


\section{Estadísticas}
A continuación se muestran gráficas estadísticas del clima en Tecate, Baja California comprendidas entre 1946 y 2012.

\begin{figure}[H]
    \centering
    \includegraphics[height=7cm]{1.png}
    \caption{Lluvia por mes}
    \label{fig:my_label}
\end{figure}

\begin{figure}[H]
    \centering
    \includegraphics[height=7cm]{2.png}
    \caption{Evaporación por mes}
    \label{fig:my_label}
\end{figure}

\begin{figure}[H]
    \centering
    \includegraphics[width=12cm]{3.png}
    \caption{Promedio y máximo de precipitación}
    \label{fig:my_label}
\end{figure}

\begin{figure}[H]
    \centering
    \includegraphics[height=8cm]{4.png}
    \caption{Promedio de lluvias diarias}
    \label{fig:my_label}
\end{figure}

\begin{figure}[H]
    \centering
    \includegraphics[height=8cm]{5.png}
    \caption{Registro diario de temperaturas máxima y mínima}
    \label{fig:my_label}
\end{figure}

\begin{figure}[H]
    \centering
    \includegraphics[width=12cm]{6.png}
    \caption{Temperaturas promedio máximas}
    \label{fig:my_label}
\end{figure}

\begin{figure}[H]
    \centering
    \includegraphics[width=12cm]{7.png}
    \caption{Temperaturas promedio mínimas}
    \label{fig:my_label}
\end{figure}

Analizando las gráficas podemos darnos cuenta que no es un lugar donde llueva mucho. Tiene varias temporadas de lluvias alrededor del año pero no son intensas ni tan frecuentes. Parece ser un lugar frió, con temperaturas promedio de 23ºC durante el verano y de 11ºC durante el invierno.

\section{Primeras Impresiones de la Actividad}
Me pareció de primer instante un poco complejo, pero ya empezando con ella fue cada vez más facil. Lo único que fue un poco complicado fue usar \LaTeX, pero sólo al principio, ya al finalizar el documento se tornó sencillo. Me parecio una actividad interesante debido a que se involucraron herramientas con las que no habia tenido tanto acercamiento y también por el tema de la misma actividad. Recomiendo que pueda ser una actividad de tema libre donde se listen los objetivos específicos de ella, supongo que en ésta fue aprender algunas cosas básicas de \LaTeX. Fuera de eso, me parecio una gran actividad y le asigno un grado de complejidad bajo.


\end{document} 
 
